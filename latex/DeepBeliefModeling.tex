\documentclass[11pt]{article}

    \usepackage[breakable]{tcolorbox}
    \usepackage{parskip} % Stop auto-indenting (to mimic markdown behaviour)

    % Basic figure setup, for now with no caption control since it's done
    % automatically by Pandoc (which extracts ![](path) syntax from Markdown).
    \usepackage{graphicx}
    % Maintain compatibility with old templates. Remove in nbconvert 6.0
    \let\Oldincludegraphics\includegraphics
    % Ensure that by default, figures have no caption (until we provide a
    % proper Figure object with a Caption API and a way to capture that
    % in the conversion process - todo).
    \usepackage{caption}
    \DeclareCaptionFormat{nocaption}{}
    \captionsetup{format=nocaption,aboveskip=0pt,belowskip=0pt}

    \usepackage{float}
    \floatplacement{figure}{H} % forces figures to be placed at the correct location
    \usepackage{xcolor} % Allow colors to be defined
    \usepackage{enumerate} % Needed for markdown enumerations to work
    \usepackage{geometry} % Used to adjust the document margins
    \usepackage{amsmath} % Equations
    \usepackage{amssymb} % Equations
    \usepackage{textcomp} % defines textquotesingle
    % Hack from http://tex.stackexchange.com/a/47451/13684:
    \AtBeginDocument{%
        \def\PYZsq{\textquotesingle}% Upright quotes in Pygmentized code
    }
    \usepackage{upquote} % Upright quotes for verbatim code
    \usepackage{eurosym} % defines \euro

    \usepackage{iftex}
    \ifPDFTeX
        \usepackage[T1]{fontenc}
        \IfFileExists{alphabeta.sty}{
              \usepackage{alphabeta}
          }{
              \usepackage[mathletters]{ucs}
              \usepackage[utf8x]{inputenc}
          }
    \else
        \usepackage{fontspec}
        \usepackage{unicode-math}
    \fi

    \usepackage{fancyvrb} % verbatim replacement that allows latex
    \usepackage{grffile} % extends the file name processing of package graphics
                         % to support a larger range
    \makeatletter % fix for old versions of grffile with XeLaTeX
    \@ifpackagelater{grffile}{2019/11/01}
    {
      % Do nothing on new versions
    }
    {
      \def\Gread@@xetex#1{%
        \IfFileExists{"\Gin@base".bb}%
        {\Gread@eps{\Gin@base.bb}}%
        {\Gread@@xetex@aux#1}%
      }
    }
    \makeatother
    \usepackage[Export]{adjustbox} % Used to constrain images to a maximum size
    \adjustboxset{max size={0.9\linewidth}{0.9\paperheight}}

    % The hyperref package gives us a pdf with properly built
    % internal navigation ('pdf bookmarks' for the table of contents,
    % internal cross-reference links, web links for URLs, etc.)
    \usepackage{hyperref}
    % The default LaTeX title has an obnoxious amount of whitespace. By default,
    % titling removes some of it. It also provides customization options.
    \usepackage{titling}
    \usepackage{longtable} % longtable support required by pandoc >1.10
    \usepackage{booktabs}  % table support for pandoc > 1.12.2
    \usepackage{array}     % table support for pandoc >= 2.11.3
    \usepackage{calc}      % table minipage width calculation for pandoc >= 2.11.1
    \usepackage[inline]{enumitem} % IRkernel/repr support (it uses the enumerate* environment)
    \usepackage[normalem]{ulem} % ulem is needed to support strikethroughs (\sout)
                                % normalem makes italics be italics, not underlines
    \usepackage{mathrsfs}
    

    
    % Colors for the hyperref package
    \definecolor{urlcolor}{rgb}{0,.145,.698}
    \definecolor{linkcolor}{rgb}{.71,0.21,0.01}
    \definecolor{citecolor}{rgb}{.12,.54,.11}

    % ANSI colors
    \definecolor{ansi-black}{HTML}{3E424D}
    \definecolor{ansi-black-intense}{HTML}{282C36}
    \definecolor{ansi-red}{HTML}{E75C58}
    \definecolor{ansi-red-intense}{HTML}{B22B31}
    \definecolor{ansi-green}{HTML}{00A250}
    \definecolor{ansi-green-intense}{HTML}{007427}
    \definecolor{ansi-yellow}{HTML}{DDB62B}
    \definecolor{ansi-yellow-intense}{HTML}{B27D12}
    \definecolor{ansi-blue}{HTML}{208FFB}
    \definecolor{ansi-blue-intense}{HTML}{0065CA}
    \definecolor{ansi-magenta}{HTML}{D160C4}
    \definecolor{ansi-magenta-intense}{HTML}{A03196}
    \definecolor{ansi-cyan}{HTML}{60C6C8}
    \definecolor{ansi-cyan-intense}{HTML}{258F8F}
    \definecolor{ansi-white}{HTML}{C5C1B4}
    \definecolor{ansi-white-intense}{HTML}{A1A6B2}
    \definecolor{ansi-default-inverse-fg}{HTML}{FFFFFF}
    \definecolor{ansi-default-inverse-bg}{HTML}{000000}

    % common color for the border for error outputs.
    \definecolor{outerrorbackground}{HTML}{FFDFDF}

    % commands and environments needed by pandoc snippets
    % extracted from the output of `pandoc -s`
    \providecommand{\tightlist}{%
      \setlength{\itemsep}{0pt}\setlength{\parskip}{0pt}}
    \DefineVerbatimEnvironment{Highlighting}{Verbatim}{commandchars=\\\{\}}
    % Add ',fontsize=\small' for more characters per line
    \newenvironment{Shaded}{}{}
    \newcommand{\KeywordTok}[1]{\textcolor[rgb]{0.00,0.44,0.13}{\textbf{{#1}}}}
    \newcommand{\DataTypeTok}[1]{\textcolor[rgb]{0.56,0.13,0.00}{{#1}}}
    \newcommand{\DecValTok}[1]{\textcolor[rgb]{0.25,0.63,0.44}{{#1}}}
    \newcommand{\BaseNTok}[1]{\textcolor[rgb]{0.25,0.63,0.44}{{#1}}}
    \newcommand{\FloatTok}[1]{\textcolor[rgb]{0.25,0.63,0.44}{{#1}}}
    \newcommand{\CharTok}[1]{\textcolor[rgb]{0.25,0.44,0.63}{{#1}}}
    \newcommand{\StringTok}[1]{\textcolor[rgb]{0.25,0.44,0.63}{{#1}}}
    \newcommand{\CommentTok}[1]{\textcolor[rgb]{0.38,0.63,0.69}{\textit{{#1}}}}
    \newcommand{\OtherTok}[1]{\textcolor[rgb]{0.00,0.44,0.13}{{#1}}}
    \newcommand{\AlertTok}[1]{\textcolor[rgb]{1.00,0.00,0.00}{\textbf{{#1}}}}
    \newcommand{\FunctionTok}[1]{\textcolor[rgb]{0.02,0.16,0.49}{{#1}}}
    \newcommand{\RegionMarkerTok}[1]{{#1}}
    \newcommand{\ErrorTok}[1]{\textcolor[rgb]{1.00,0.00,0.00}{\textbf{{#1}}}}
    \newcommand{\NormalTok}[1]{{#1}}

    % Additional commands for more recent versions of Pandoc
    \newcommand{\ConstantTok}[1]{\textcolor[rgb]{0.53,0.00,0.00}{{#1}}}
    \newcommand{\SpecialCharTok}[1]{\textcolor[rgb]{0.25,0.44,0.63}{{#1}}}
    \newcommand{\VerbatimStringTok}[1]{\textcolor[rgb]{0.25,0.44,0.63}{{#1}}}
    \newcommand{\SpecialStringTok}[1]{\textcolor[rgb]{0.73,0.40,0.53}{{#1}}}
    \newcommand{\ImportTok}[1]{{#1}}
    \newcommand{\DocumentationTok}[1]{\textcolor[rgb]{0.73,0.13,0.13}{\textit{{#1}}}}
    \newcommand{\AnnotationTok}[1]{\textcolor[rgb]{0.38,0.63,0.69}{\textbf{\textit{{#1}}}}}
    \newcommand{\CommentVarTok}[1]{\textcolor[rgb]{0.38,0.63,0.69}{\textbf{\textit{{#1}}}}}
    \newcommand{\VariableTok}[1]{\textcolor[rgb]{0.10,0.09,0.49}{{#1}}}
    \newcommand{\ControlFlowTok}[1]{\textcolor[rgb]{0.00,0.44,0.13}{\textbf{{#1}}}}
    \newcommand{\OperatorTok}[1]{\textcolor[rgb]{0.40,0.40,0.40}{{#1}}}
    \newcommand{\BuiltInTok}[1]{{#1}}
    \newcommand{\ExtensionTok}[1]{{#1}}
    \newcommand{\PreprocessorTok}[1]{\textcolor[rgb]{0.74,0.48,0.00}{{#1}}}
    \newcommand{\AttributeTok}[1]{\textcolor[rgb]{0.49,0.56,0.16}{{#1}}}
    \newcommand{\InformationTok}[1]{\textcolor[rgb]{0.38,0.63,0.69}{\textbf{\textit{{#1}}}}}
    \newcommand{\WarningTok}[1]{\textcolor[rgb]{0.38,0.63,0.69}{\textbf{\textit{{#1}}}}}


    % Define a nice break command that doesn't care if a line doesn't already
    % exist.
    \def\br{\hspace*{\fill} \\* }
    % Math Jax compatibility definitions
    \def\gt{>}
    \def\lt{<}
    \let\Oldtex\TeX
    \let\Oldlatex\LaTeX
    \renewcommand{\TeX}{\textrm{\Oldtex}}
    \renewcommand{\LaTeX}{\textrm{\Oldlatex}}
    % Document parameters
    % Document title
    \title{Modelowanie Głębokich Przekonań
    przy użyciu sieci neuronowych}
    \author{Jakub Dajczak, Miłosz Chojnacki, Anton Delinac}
    
    
    
    
    
% Pygments definitions
\makeatletter
\def\PY@reset{\let\PY@it=\relax \let\PY@bf=\relax%
    \let\PY@ul=\relax \let\PY@tc=\relax%
    \let\PY@bc=\relax \let\PY@ff=\relax}
\def\PY@tok#1{\csname PY@tok@#1\endcsname}
\def\PY@toks#1+{\ifx\relax#1\empty\else%
    \PY@tok{#1}\expandafter\PY@toks\fi}
\def\PY@do#1{\PY@bc{\PY@tc{\PY@ul{%
    \PY@it{\PY@bf{\PY@ff{#1}}}}}}}
\def\PY#1#2{\PY@reset\PY@toks#1+\relax+\PY@do{#2}}

\@namedef{PY@tok@w}{\def\PY@tc##1{\textcolor[rgb]{0.73,0.73,0.73}{##1}}}
\@namedef{PY@tok@c}{\let\PY@it=\textit\def\PY@tc##1{\textcolor[rgb]{0.24,0.48,0.48}{##1}}}
\@namedef{PY@tok@cp}{\def\PY@tc##1{\textcolor[rgb]{0.61,0.40,0.00}{##1}}}
\@namedef{PY@tok@k}{\let\PY@bf=\textbf\def\PY@tc##1{\textcolor[rgb]{0.00,0.50,0.00}{##1}}}
\@namedef{PY@tok@kp}{\def\PY@tc##1{\textcolor[rgb]{0.00,0.50,0.00}{##1}}}
\@namedef{PY@tok@kt}{\def\PY@tc##1{\textcolor[rgb]{0.69,0.00,0.25}{##1}}}
\@namedef{PY@tok@o}{\def\PY@tc##1{\textcolor[rgb]{0.40,0.40,0.40}{##1}}}
\@namedef{PY@tok@ow}{\let\PY@bf=\textbf\def\PY@tc##1{\textcolor[rgb]{0.67,0.13,1.00}{##1}}}
\@namedef{PY@tok@nb}{\def\PY@tc##1{\textcolor[rgb]{0.00,0.50,0.00}{##1}}}
\@namedef{PY@tok@nf}{\def\PY@tc##1{\textcolor[rgb]{0.00,0.00,1.00}{##1}}}
\@namedef{PY@tok@nc}{\let\PY@bf=\textbf\def\PY@tc##1{\textcolor[rgb]{0.00,0.00,1.00}{##1}}}
\@namedef{PY@tok@nn}{\let\PY@bf=\textbf\def\PY@tc##1{\textcolor[rgb]{0.00,0.00,1.00}{##1}}}
\@namedef{PY@tok@ne}{\let\PY@bf=\textbf\def\PY@tc##1{\textcolor[rgb]{0.80,0.25,0.22}{##1}}}
\@namedef{PY@tok@nv}{\def\PY@tc##1{\textcolor[rgb]{0.10,0.09,0.49}{##1}}}
\@namedef{PY@tok@no}{\def\PY@tc##1{\textcolor[rgb]{0.53,0.00,0.00}{##1}}}
\@namedef{PY@tok@nl}{\def\PY@tc##1{\textcolor[rgb]{0.46,0.46,0.00}{##1}}}
\@namedef{PY@tok@ni}{\let\PY@bf=\textbf\def\PY@tc##1{\textcolor[rgb]{0.44,0.44,0.44}{##1}}}
\@namedef{PY@tok@na}{\def\PY@tc##1{\textcolor[rgb]{0.41,0.47,0.13}{##1}}}
\@namedef{PY@tok@nt}{\let\PY@bf=\textbf\def\PY@tc##1{\textcolor[rgb]{0.00,0.50,0.00}{##1}}}
\@namedef{PY@tok@nd}{\def\PY@tc##1{\textcolor[rgb]{0.67,0.13,1.00}{##1}}}
\@namedef{PY@tok@s}{\def\PY@tc##1{\textcolor[rgb]{0.73,0.13,0.13}{##1}}}
\@namedef{PY@tok@sd}{\let\PY@it=\textit\def\PY@tc##1{\textcolor[rgb]{0.73,0.13,0.13}{##1}}}
\@namedef{PY@tok@si}{\let\PY@bf=\textbf\def\PY@tc##1{\textcolor[rgb]{0.64,0.35,0.47}{##1}}}
\@namedef{PY@tok@se}{\let\PY@bf=\textbf\def\PY@tc##1{\textcolor[rgb]{0.67,0.36,0.12}{##1}}}
\@namedef{PY@tok@sr}{\def\PY@tc##1{\textcolor[rgb]{0.64,0.35,0.47}{##1}}}
\@namedef{PY@tok@ss}{\def\PY@tc##1{\textcolor[rgb]{0.10,0.09,0.49}{##1}}}
\@namedef{PY@tok@sx}{\def\PY@tc##1{\textcolor[rgb]{0.00,0.50,0.00}{##1}}}
\@namedef{PY@tok@m}{\def\PY@tc##1{\textcolor[rgb]{0.40,0.40,0.40}{##1}}}
\@namedef{PY@tok@gh}{\let\PY@bf=\textbf\def\PY@tc##1{\textcolor[rgb]{0.00,0.00,0.50}{##1}}}
\@namedef{PY@tok@gu}{\let\PY@bf=\textbf\def\PY@tc##1{\textcolor[rgb]{0.50,0.00,0.50}{##1}}}
\@namedef{PY@tok@gd}{\def\PY@tc##1{\textcolor[rgb]{0.63,0.00,0.00}{##1}}}
\@namedef{PY@tok@gi}{\def\PY@tc##1{\textcolor[rgb]{0.00,0.52,0.00}{##1}}}
\@namedef{PY@tok@gr}{\def\PY@tc##1{\textcolor[rgb]{0.89,0.00,0.00}{##1}}}
\@namedef{PY@tok@ge}{\let\PY@it=\textit}
\@namedef{PY@tok@gs}{\let\PY@bf=\textbf}
\@namedef{PY@tok@gp}{\let\PY@bf=\textbf\def\PY@tc##1{\textcolor[rgb]{0.00,0.00,0.50}{##1}}}
\@namedef{PY@tok@go}{\def\PY@tc##1{\textcolor[rgb]{0.44,0.44,0.44}{##1}}}
\@namedef{PY@tok@gt}{\def\PY@tc##1{\textcolor[rgb]{0.00,0.27,0.87}{##1}}}
\@namedef{PY@tok@err}{\def\PY@bc##1{{\setlength{\fboxsep}{\string -\fboxrule}\fcolorbox[rgb]{1.00,0.00,0.00}{1,1,1}{\strut ##1}}}}
\@namedef{PY@tok@kc}{\let\PY@bf=\textbf\def\PY@tc##1{\textcolor[rgb]{0.00,0.50,0.00}{##1}}}
\@namedef{PY@tok@kd}{\let\PY@bf=\textbf\def\PY@tc##1{\textcolor[rgb]{0.00,0.50,0.00}{##1}}}
\@namedef{PY@tok@kn}{\let\PY@bf=\textbf\def\PY@tc##1{\textcolor[rgb]{0.00,0.50,0.00}{##1}}}
\@namedef{PY@tok@kr}{\let\PY@bf=\textbf\def\PY@tc##1{\textcolor[rgb]{0.00,0.50,0.00}{##1}}}
\@namedef{PY@tok@bp}{\def\PY@tc##1{\textcolor[rgb]{0.00,0.50,0.00}{##1}}}
\@namedef{PY@tok@fm}{\def\PY@tc##1{\textcolor[rgb]{0.00,0.00,1.00}{##1}}}
\@namedef{PY@tok@vc}{\def\PY@tc##1{\textcolor[rgb]{0.10,0.09,0.49}{##1}}}
\@namedef{PY@tok@vg}{\def\PY@tc##1{\textcolor[rgb]{0.10,0.09,0.49}{##1}}}
\@namedef{PY@tok@vi}{\def\PY@tc##1{\textcolor[rgb]{0.10,0.09,0.49}{##1}}}
\@namedef{PY@tok@vm}{\def\PY@tc##1{\textcolor[rgb]{0.10,0.09,0.49}{##1}}}
\@namedef{PY@tok@sa}{\def\PY@tc##1{\textcolor[rgb]{0.73,0.13,0.13}{##1}}}
\@namedef{PY@tok@sb}{\def\PY@tc##1{\textcolor[rgb]{0.73,0.13,0.13}{##1}}}
\@namedef{PY@tok@sc}{\def\PY@tc##1{\textcolor[rgb]{0.73,0.13,0.13}{##1}}}
\@namedef{PY@tok@dl}{\def\PY@tc##1{\textcolor[rgb]{0.73,0.13,0.13}{##1}}}
\@namedef{PY@tok@s2}{\def\PY@tc##1{\textcolor[rgb]{0.73,0.13,0.13}{##1}}}
\@namedef{PY@tok@sh}{\def\PY@tc##1{\textcolor[rgb]{0.73,0.13,0.13}{##1}}}
\@namedef{PY@tok@s1}{\def\PY@tc##1{\textcolor[rgb]{0.73,0.13,0.13}{##1}}}
\@namedef{PY@tok@mb}{\def\PY@tc##1{\textcolor[rgb]{0.40,0.40,0.40}{##1}}}
\@namedef{PY@tok@mf}{\def\PY@tc##1{\textcolor[rgb]{0.40,0.40,0.40}{##1}}}
\@namedef{PY@tok@mh}{\def\PY@tc##1{\textcolor[rgb]{0.40,0.40,0.40}{##1}}}
\@namedef{PY@tok@mi}{\def\PY@tc##1{\textcolor[rgb]{0.40,0.40,0.40}{##1}}}
\@namedef{PY@tok@il}{\def\PY@tc##1{\textcolor[rgb]{0.40,0.40,0.40}{##1}}}
\@namedef{PY@tok@mo}{\def\PY@tc##1{\textcolor[rgb]{0.40,0.40,0.40}{##1}}}
\@namedef{PY@tok@ch}{\let\PY@it=\textit\def\PY@tc##1{\textcolor[rgb]{0.24,0.48,0.48}{##1}}}
\@namedef{PY@tok@cm}{\let\PY@it=\textit\def\PY@tc##1{\textcolor[rgb]{0.24,0.48,0.48}{##1}}}
\@namedef{PY@tok@cpf}{\let\PY@it=\textit\def\PY@tc##1{\textcolor[rgb]{0.24,0.48,0.48}{##1}}}
\@namedef{PY@tok@c1}{\let\PY@it=\textit\def\PY@tc##1{\textcolor[rgb]{0.24,0.48,0.48}{##1}}}
\@namedef{PY@tok@cs}{\let\PY@it=\textit\def\PY@tc##1{\textcolor[rgb]{0.24,0.48,0.48}{##1}}}

\def\PYZbs{\char`\\}
\def\PYZus{\char`\_}
\def\PYZob{\char`\{}
\def\PYZcb{\char`\}}
\def\PYZca{\char`\^}
\def\PYZam{\char`\&}
\def\PYZlt{\char`\<}
\def\PYZgt{\char`\>}
\def\PYZsh{\char`\#}
\def\PYZpc{\char`\%}
\def\PYZdl{\char`\$}
\def\PYZhy{\char`\-}
\def\PYZsq{\char`\'}
\def\PYZdq{\char`\"}
\def\PYZti{\char`\~}
% for compatibility with earlier versions
\def\PYZat{@}
\def\PYZlb{[}
\def\PYZrb{]}
\makeatother


    % For linebreaks inside Verbatim environment from package fancyvrb.
    \makeatletter
        \newbox\Wrappedcontinuationbox
        \newbox\Wrappedvisiblespacebox
        \newcommand*\Wrappedvisiblespace {\textcolor{red}{\textvisiblespace}}
        \newcommand*\Wrappedcontinuationsymbol {\textcolor{red}{\llap{\tiny$\m@th\hookrightarrow$}}}
        \newcommand*\Wrappedcontinuationindent {3ex }
        \newcommand*\Wrappedafterbreak {\kern\Wrappedcontinuationindent\copy\Wrappedcontinuationbox}
        % Take advantage of the already applied Pygments mark-up to insert
        % potential linebreaks for TeX processing.
        %        {, <, #, %, $, ' and ": go to next line.
        %        _, }, ^, &, >, - and ~: stay at end of broken line.
        % Use of \textquotesingle for straight quote.
        \newcommand*\Wrappedbreaksatspecials {%
            \def\PYGZus{\discretionary{\char`\_}{\Wrappedafterbreak}{\char`\_}}%
            \def\PYGZob{\discretionary{}{\Wrappedafterbreak\char`\{}{\char`\{}}%
            \def\PYGZcb{\discretionary{\char`\}}{\Wrappedafterbreak}{\char`\}}}%
            \def\PYGZca{\discretionary{\char`\^}{\Wrappedafterbreak}{\char`\^}}%
            \def\PYGZam{\discretionary{\char`\&}{\Wrappedafterbreak}{\char`\&}}%
            \def\PYGZlt{\discretionary{}{\Wrappedafterbreak\char`\<}{\char`\<}}%
            \def\PYGZgt{\discretionary{\char`\>}{\Wrappedafterbreak}{\char`\>}}%
            \def\PYGZsh{\discretionary{}{\Wrappedafterbreak\char`\#}{\char`\#}}%
            \def\PYGZpc{\discretionary{}{\Wrappedafterbreak\char`\%}{\char`\%}}%
            \def\PYGZdl{\discretionary{}{\Wrappedafterbreak\char`\$}{\char`\$}}%
            \def\PYGZhy{\discretionary{\char`\-}{\Wrappedafterbreak}{\char`\-}}%
            \def\PYGZsq{\discretionary{}{\Wrappedafterbreak\textquotesingle}{\textquotesingle}}%
            \def\PYGZdq{\discretionary{}{\Wrappedafterbreak\char`\"}{\char`\"}}%
            \def\PYGZti{\discretionary{\char`\~}{\Wrappedafterbreak}{\char`\~}}%
        }
        % Some characters . , ; ? ! / are not pygmentized.
        % This macro makes them "active" and they will insert potential linebreaks
        \newcommand*\Wrappedbreaksatpunct {%
            \lccode`\~`\.\lowercase{\def~}{\discretionary{\hbox{\char`\.}}{\Wrappedafterbreak}{\hbox{\char`\.}}}%
            \lccode`\~`\,\lowercase{\def~}{\discretionary{\hbox{\char`\,}}{\Wrappedafterbreak}{\hbox{\char`\,}}}%
            \lccode`\~`\;\lowercase{\def~}{\discretionary{\hbox{\char`\;}}{\Wrappedafterbreak}{\hbox{\char`\;}}}%
            \lccode`\~`\:\lowercase{\def~}{\discretionary{\hbox{\char`\:}}{\Wrappedafterbreak}{\hbox{\char`\:}}}%
            \lccode`\~`\?\lowercase{\def~}{\discretionary{\hbox{\char`\?}}{\Wrappedafterbreak}{\hbox{\char`\?}}}%
            \lccode`\~`\!\lowercase{\def~}{\discretionary{\hbox{\char`\!}}{\Wrappedafterbreak}{\hbox{\char`\!}}}%
            \lccode`\~`\/\lowercase{\def~}{\discretionary{\hbox{\char`\/}}{\Wrappedafterbreak}{\hbox{\char`\/}}}%
            \catcode`\.\active
            \catcode`\,\active
            \catcode`\;\active
            \catcode`\:\active
            \catcode`\?\active
            \catcode`\!\active
            \catcode`\/\active
            \lccode`\~`\~
        }
    \makeatother

    \let\OriginalVerbatim=\Verbatim
    \makeatletter
    \renewcommand{\Verbatim}[1][1]{%
        %\parskip\z@skip
        \sbox\Wrappedcontinuationbox {\Wrappedcontinuationsymbol}%
        \sbox\Wrappedvisiblespacebox {\FV@SetupFont\Wrappedvisiblespace}%
        \def\FancyVerbFormatLine ##1{\hsize\linewidth
            \vtop{\raggedright\hyphenpenalty\z@\exhyphenpenalty\z@
                \doublehyphendemerits\z@\finalhyphendemerits\z@
                \strut ##1\strut}%
        }%
        % If the linebreak is at a space, the latter will be displayed as visible
        % space at end of first line, and a continuation symbol starts next line.
        % Stretch/shrink are however usually zero for typewriter font.
        \def\FV@Space {%
            \nobreak\hskip\z@ plus\fontdimen3\font minus\fontdimen4\font
            \discretionary{\copy\Wrappedvisiblespacebox}{\Wrappedafterbreak}
            {\kern\fontdimen2\font}%
        }%

        % Allow breaks at special characters using \PYG... macros.
        \Wrappedbreaksatspecials
        % Breaks at punctuation characters . , ; ? ! and / need catcode=\active
        \OriginalVerbatim[#1,codes*=\Wrappedbreaksatpunct]%
    }
    \makeatother

    % Exact colors from NB
    \definecolor{incolor}{HTML}{303F9F}
    \definecolor{outcolor}{HTML}{D84315}
    \definecolor{cellborder}{HTML}{CFCFCF}
    \definecolor{cellbackground}{HTML}{F7F7F7}

    % prompt
    \makeatletter
    \newcommand{\boxspacing}{\kern\kvtcb@left@rule\kern\kvtcb@boxsep}
    \makeatother
    \newcommand{\prompt}[4]{
        {\ttfamily\llap{{\color{#2}[#3]:\hspace{3pt}#4}}\vspace{-\baselineskip}}
    }
    

    
    % Prevent overflowing lines due to hard-to-break entities
    \sloppy
    % Setup hyperref package
    \hypersetup{
      breaklinks=true,  % so long urls are correctly broken across lines
      colorlinks=true,
      urlcolor=urlcolor,
      linkcolor=linkcolor,
      citecolor=citecolor,
      }
    % Slightly bigger margins than the latex defaults
    
    \geometry{verbose,tmargin=1in,bmargin=1in,lmargin=1in,rmargin=1in}
    
    

\begin{document}
    
    \maketitle
    

\section*{Wstęp}

Celem niniejszego projektu było przygotowanie oprogramowania umożliwiającego 
analizę głębokich przekonań człowieka oraz tego jak wpływa na nie dostęp do 
nowych informacji. W naszym przypadku przyjęte strony to osoby popierające 
szczepienia na covid-19 i osoby będące przeciwko tym szczepieniom. W dalszej 
części sprawozdania będziemy te dwie grupy nazywać kolejno "proszczepionką" 
oraz "antyszczepionkową".

\subsection*{Wykorzystane biblioteki}

Do pracy przy projekcie posługiwaliśmy się głównie bibloteką sklearn która posłużyła nam do
przygotowania danych i klasyfikacji oraz matplotlib służącą do wizualizacji danych.
Sieć samo-organizującą wykorzystaliśmy z modułu MiniSOM.

Dodatkowo stosowaliśmy wiele stworzonych przez nas funkcji, które pozwoliły nam 
ułatwić pracę nad projektem.

\section*{Dane}

Dane do naszego projektu to artykuły, które pozyskaliśmy z artykułów naukowych 
lub pseudonaukowych oraz
z publicznych for gdzie użytkownicy dobrowolnie upubliczniali swoje opinie 
na temat szczepień covid-19.

Liczba tych artykułów wynosi 340, na które składają się 170 
artykułów grupy pro i 170 artykułów grupy anty. Dane zestawiliśmy w pojedynczym 
pliku .csv, który wczytaliśmy za pomocą biblioteki pandas. W dalszej części 
sprawozdania artykuły antyszczepionkowe są oznaczane liczbą 0, a artykuły 
proszczepionkowe liczbą 1.

        
\begin{center}
\adjustimage{max size={0.9\linewidth}{0.9\paperheight}}{DeepBeliefModeling_files/DeepBeliefModeling_6_1.png}
\end{center}
    
    
\subsection*{Przygotowanie danych do klasyfikacji}
 
Z wykorzystaniem biblioteki nltk oczyszczaliśmy artykuły usuwając z nich 
znaki interpunkcyjne, znaki białe, linki i "stop words". Dodatkowo zmienialiśmy
wszystkie litery na małe litery.

Dodatkowo, próbowaliśmy szukać par słów, które występują w artykułach, 
bazując na już przygotowanych listach słów, jednak nie przyniosło to 
satysfakcjonujących wyników.

Przykładowe 10 pierwszych artykułów z grupy antyszczepionkowej:

    \begin{Verbatim}[commandchars=\\\{\}]
0    It's been quite a week. I've been in the media{\ldots}
1    This week has been eye-opening. Even for me --{\ldots}
2    Yes, it's true. I beat COVID-19 in 48 hours wi{\ldots}
3    The COVID-19 vaccines appear to be causing a g{\ldots}
4    Back in the 1980s, I was a Columbia University{\ldots}
5    So you got the vaccine because they told you t{\ldots}
6    I Am A Living Proof That COVID-19 Is Fake This{\ldots}
7    I have a PhD in virology and immunology. I’m a{\ldots}
8    How do you convince the world’s population to {\ldots}
9    Take 15 minutes and listen to this interview w{\ldots}
Name: data, dtype: object
0    [quite, week, medium, business, decade, never,{\ldots}
1    [week, eye, opening, even, guy, warned, year, {\ldots}
2    [yes, true, beat, covid, hour, ivermectin, get{\ldots}
3    [covid, vaccine, appear, causing, global, heal{\ldots}
4    [back, columbia, university, student, learning{\ldots}
5    [got, vaccine, told, get, forced, get, joe, bi{\ldots}
6    [living, proof, covid, fake, unfiltered, take,{\ldots}
7    [phd, virology, immunology, clinical, lab, sci{\ldots}
8    [convince, world, population, take, unproven, {\ldots}
9    [take, minute, listen, interview, hospital, nu{\ldots}
Name: data, dtype: object
    \end{Verbatim}

\section*{Klasyfikacja artykułów}

\subsection*{Wektoryzacja danych}

Następnym zadaniem było przygotowanie odpowiedniej klasyfikacji artykułów.
Postanowiliśmy wybrać jeden z modeli przestrzeni wektorowych (SVM) - "Worek słów"
(Bag of Words). Do wyznaczenia wektorów użyliśmy biblioteki sklearn, a z niej 
gotowego rozwiązania TfidfVectorizer.

Jest to bardzo popularny algorytm do transformacji tekstu na wartości liczbowe,
służące do uczenia maszynowego. TfidfVectorizer jest połączeniem TfidfTransformer
i CountVectorizer. CountVectorizer korzystając z przygotowanej przez nas 
funkcji generuje macierz, która zawiera liczby wystąpień słów w artykułach. 
TfidfTransformer jest oparty o
TF-IDF - Term Frequency Inverse Document Frequency.

\[ TFIDF = TF(t,d) * IDF(t) \]

gdzie:
\begin{itemize}
    \item TF(t,d) - liczba wystąpień słowa t w artykule d
    \item IDF(t) - logarytm liczby artykułów w których wystąpił słowo t: \( \log\frac{n}{1+df(t,d)} \)
\end{itemize}

Wygląd naszej macierzy Bag of Words:

\begin{Verbatim}[commandchars=\\\{\}]
    ['aabduzrw' 'aaby' 'aadhaar' {\ldots} 'zoster' 'zuckerberg' 'zurich']
    Tf-idf vectorizer
         aabduzrw  aaby    {\ldots}   zuckerberg  zurich
    0         0.0   0.0    {\ldots}         0.0     0.0
    1         0.0   0.0    {\ldots}         0.0     0.0
    2         0.0   0.0    {\ldots}         0.0     0.0
    3         0.0   0.0    {\ldots}         0.0     0.0
    4         0.0   0.0    {\ldots}         0.0     0.0
    ..      {\ldots}    {\ldots}    {\ldots}        {\ldots}     {\ldots}
    335       0.0   0.0    {\ldots}         0.0     0.0
    336       0.0   0.0    {\ldots}         0.0     0.0
    337       0.0   0.0    {\ldots}         0.0     0.0
    338       0.0   0.0    {\ldots}         0.0     0.0
    339       0.0   0.0    {\ldots}         0.0     0.0   
         
    [340 rows x 14698 columns]
\end{Verbatim}
    

\subsection*{Klasyfikacja}

Wstępnie naszym głównym klasifkatorem był klasyfikator Bayesa - Naive Bayes.
Działał on na poziomie skuteczności 70\% - 80\%. W przypadku naszego projektu
jednak zdecydowaliśmy się na klasyfikator SVM, którego skuteczność wynosiła 
ponad 90\%. SVM za pomocą funkcji kernelowych - w naszym przypadku funkcji liniowej
- oblicza relacje między poszczególnymi artykułami i stara 
się dopasować płaszczyznę w wielowymiarowej przestrzeni tak, aby jak najlepiej 
oddzielała ona artykuły jednej klasy od drugiej. Odległości punktów od wyznaczonej 
płaszczyzny nazywamy wektorami wspierającymi, a marginesem nazywamy odległość 
najbliższych punktów od płaszczyzny dzielącej.

Ostatecznie użyliśmy modelu SGDClassifier z biblioteki sklearn.
Jest to wersja algorytmu SVM z wstępnym treningiem SGD.

\subsection*{Wizualizacja}

Na potrzeby wizualizacji przyjeliśmy, że grupa artykułów antyszczepionkowych 
będzie oznaczona kolorem czerwonym, a grupa proszczepionkowa - zielonym.

Wykorzystując PCA (Principal component analysis), klasyfikator SVC oraz 
LSA (Latent semantic analysis), rzutujemy artykuły na przestrzeń dwuwymiarową.

TruncatedSVD - implementacja algorytmu LSA z biblioteki sklearn.
Podaliśmy na wejście algorytmu macierz Bag of Words, aby uzysać kordynaty artykułów w przestzreni.

    \begin{center}
    \adjustimage{max size={0.9\linewidth}{0.9\paperheight}}{DeepBeliefModeling_files/DeepBeliefModeling_13_0.png}
    \end{center}
    { \hspace*{\fill} \\}

PCA - implementacja algorytmu PCA z biblioteki sklearn. Redukcja wymiarów macierzy w oparciu o
wzajemne podobnieństwa między artykułami.

SVC - Support Vector Classification. Również z implementacji sklearn. Na pierwszym wykresie
korzystając z liniowej klasyfikacji, na drugim z nieliniowej.
    
    \begin{center}
    \adjustimage{max size={0.9\linewidth}{0.9\paperheight}}{DeepBeliefModeling_files/DeepBeliefModeling_15_1.png}
    \end{center}
    { \hspace*{\fill} \\}
    
    \begin{center}
    \adjustimage{max size={0.9\linewidth}{0.9\paperheight}}{DeepBeliefModeling_files/DeepBeliefModeling_14_0.png}
    \end{center}
    { \hspace*{\fill} \\}

    Jak widać na powyższych wykresach granica między artykułami antyszczepionkowymi,
    a proszczepionkowymi jest dobrze widoczna, jednak margines obejmuje niektóre
    artykuły, z klasy przeciwnej.

    Wizualizacja za pomocą PCA i SVC jest bardzej czytelna, ale to głównie dzięki
    zastosowaniu kolorów tła i lini hiperpłaszczyzny.
    
\subsection*{Wyznaczenie słów kluczowych}

Do wyznaczenia słów kluczowych wykorzystaliśmy wyżej już wspominany klasyfikator SGDClassifier, 
który zapisuje cechy danej klasy oraz odpowiadające im wagi. Napisaliśmy funkcję 
do pozyskania tych kluczowych słów w oparciu właśnie o wagi klasyfikatora. Dla każdej klasy
uzyskaliśmy po 50 słów kluczowych.

\begin{table}[H]
    \begin{tabular}{|l|l|l|}
       & PRO               & ANTI          \\
    1  & barré             & injury        \\
    2  & vaccinated        & report        \\
    3  & safer             & tinnitus      \\
    4  & worker            & adverse       \\
    5  & cell              & mass          \\
    6  & protection        & dobbs         \\
    7  & getting           & injection     \\
    8  & immunocompromised & government    \\
    9  & additional        & ivermectin    \\
    10 & oxford            & mandate       \\
    11 & horo              & wuhan         \\
    12 & might             & jab           \\
    13 & omicron           & death         \\
    14 & le                & kill          \\
    15 & candidate         & lymphocyte    \\
    16 & seroconversion    & wrote         \\
    17 & two               & israel        \\
    18 & volunteer         & medium        \\
    19 & clinical          & vanden        \\
    20 & illinois          & article       \\
    21 & restaurant        & hour          \\
    22 & travel            & fauci         \\
    23 & biontech          & autoimmune    \\
    24 & protect           & pilot         \\
    25 & infected          & lie           \\
    26 & monitoring        & bossche       \\
    27 & rare              & natural       \\
    28 & approval          & ulcer         \\
    29 & placebo           & batch         \\
    30 & morris            & fraiman       \\
    31 & side              & product       \\
    32 & booster           & lied          \\
    33 & lower             & experimental  \\
    34 & people            & harrington    \\
    35 & chance            & paper         \\
    36 & corbett           & fraud         \\
    37 & participant       & biden         \\
    38 & moderna           & doctor        \\
    39 & safety            & snake         \\
    40 & antigen           & innate        \\
    41 & dose              & immunity      \\
    42 & sarscov           & order         \\
    43 & efficacy          & marburg       \\
    44 & effective         & dore          \\
    45 & dos               & aid           \\
    46 & part              & evidence      \\
    47 & risk              & joe           \\
    48 & trial             & patent        \\
    49 & get               & depopulation  \\
    50 & vaccine           & native       
    \end{tabular}
\end{table}
    
\section*{Sieć samo-organizująca}

Mapa Kohonena zwana również siecią neuronową Kohonena jest najbardziej znaną i najczęściej 
stosowaną siecią samouczącą się, realizującą zasadę samoorganizacji (SOM). Sieć ta 
cechuje się tym, że działaja w wielowymiarowych przestrzeniach danych wejściowych, 
w związku z czym warstwa wejściowa zawiera bardzo wiele neuronów. Proces uczenia 
można podzielić na kilka etapów:
\begin{enumerate}
    \item Inicjalizacja sieci wagami
    \item Wylosowanie danych wejściowych
    \item Wybranie wygranego neuronu na bazie odległości cosinusowej (często używa się także odl. euklidesowej)
    \item Aktualizacja wag neuronów
    \item Powrót do kroku drugiego 
\end{enumerate}

Bardzo ważnym krokiem jest aktualizacja wag neuronów. Wartość o jaką aktualizujemy 
dany neuron zależy od współczynnika uczenia (learning rate) oraz sąsiedztwa topologicznego
z innymi słowami. W dalszej części sprawozdania będziemy analizować wpływ learning rate’u
na proces uczenia.

\subsection*{Wektory artykułów}

Na wstępnym etepie pracy postanowiliśmy zwizualizować nasze artykuły w postaci wektorów słów kluczowych.
Tym sposobem pozbywamy się wielu nieistotnych informacji, które nie są potrzebne do analizy.
Dla każdego z artykułów zliczamy ilość wystąpień każdego słowa kluczowego. Następnie, dla 
normalizacji wyników używamy TfidfTransformer - jednego z etapów już wyżej wspominanego TfidfVectorizera.

Tak przygotowane wektory używamy tak samo jak przy wizualizacji artykułów przy pomocy PCA i SVC.
Efekt jest o wiele lepszy niż w używania pełnych wektorów.
W dalszym etapie inicjalizujemy mapę Kohonena i podajemy wektory artykułów do nauki.
Po ukończeniu etapu uczenia wyświetlamy wyniki w postaci wykresów.

    \begin{center}
    \adjustimage{max size={0.9\linewidth}{0.9\paperheight}}{DeepBeliefModeling_files/DeepBeliefModeling_19_1.png}
    \end{center}
    { \hspace*{\fill} \\}

    \begin{center}
    \adjustimage{max size={0.9\linewidth}{0.9\paperheight}}{DeepBeliefModeling_files/DeepBeliefModeling_18_1.png}
    \end{center}
    { \hspace*{\fill} \\}
    
\subsection*{Wektory słów kluczowych}

Dalszą częścią zadania była wizualizacja wektorów słów kluczowych w zależności od ich sąsiedztwa.
Napisaliśmy więc funkcję zliczającą wystąpienia sąsiadujących słów dla każdego 
słowa kluczowego oraz funkcję generującą słownik, gdzie kluczem jest słowo kluczowe, 
a wartością słownik słów sąsiadujących z liczbami wystąpień tych słów jako wartości.
Z tego też przy podaniu odpowiedniego sąsiedztwa uzyskiwaliśmy wektory słów kluczowych.

\subsection*{Wektory słów kluczowych - symulacja poglądów}

Następną częścią zadania była symulacja poglądów. W tym etapie używamy 
generatora wektorów słów kluczowych, dla którego stopniowo zwiększamy zasób słownictwa który
odpowiada kolejnym przeczytanym artykułom. Wybór artykułów jest losowy i jest warunkowany jedynie
zbiorem jaki podamy na wejście.

Badamy 4 możliwe scenariusze:
\begin{itemize}
    \item Osoba wierząca w teorię antyszczepionkową zaczyna czytać artykuły zachęcające do szczepień;
    \item Osoba przekonana co do skuteczności szczepionek zaczyna czytać artykuły przeciwko 
    szczepieniom;
    \item Osoba niezdecydowana zaczyna czytać artykuły zachęcające do szczepień;
    \item Osoba niezdecydowana zaczyna czytać artykuły przeciwko szczepieniom.
\end{itemize}

Dla każdej z trzech osób wstępnie przygotowujemy mapę kohonena oraz uczymy ją 80\% wszystkich artykułów z danego zbioru.
Przygotowane w mapy zwizualizowane są poniżej:

    \subsubsection*{Osoba wierząca w teorię antyszczepionkową}

    \begin{center}
    \adjustimage{max size={0.9\linewidth}{0.9\paperheight}}{DeepBeliefModeling_files/DeepBeliefModeling_29_1.png}
    \end{center}
    { \hspace*{\fill} \\}

    \subsubsection*{Osoba przekonana co do skuteczności szczepionek}

    \begin{center}
    \adjustimage{max size={0.9\linewidth}{0.9\paperheight}}{DeepBeliefModeling_files/DeepBeliefModeling_30_1.png}
    \end{center}
    { \hspace*{\fill} \\}

    \subsubsection*{Osoba niezdecydowana}

    \begin{center}
    \adjustimage{max size={0.9\linewidth}{0.9\paperheight}}{DeepBeliefModeling_files/DeepBeliefModeling_31_1.png}
    \end{center}
    { \hspace*{\fill} \\}

W dalszej części sprawozdania znajdują się wykresy obrazujące wyniki symulacji.
Dla każdego przypadku jest przedstawione 7 wykresów. Każdy kolejny wykres jest symulacją osoby
z inną podatnością na przeczytane informacje. Pierwszy wykres to symulacja osoby łatwowiernej
i szybko zmieniającej poglądy, a każdy następny wykres odpowiada osobie, która jest coraz mniej 
otwarta na nowe informacje. 

\newpage
\subsubsection*{Osoba antyszczepionkowa czyta artykuły zachęcające do szczepień}

    Learning rate = 1.0

    \begin{center}
    \adjustimage{max size={0.9\linewidth}{0.9\paperheight}}{DeepBeliefModeling_files/DeepBeliefModeling_29_3.png}
    \end{center}
    { \hspace*{\fill} \\}

    \newpage
    Learning rate = 0.5

    \begin{center}
    \adjustimage{max size={0.9\linewidth}{0.9\paperheight}}{DeepBeliefModeling_files/DeepBeliefModeling_29_5.png}
    \end{center}
    { \hspace*{\fill} \\}

    \newpage
    Learning rate = 0.25

    \begin{center}
    \adjustimage{max size={0.9\linewidth}{0.9\paperheight}}{DeepBeliefModeling_files/DeepBeliefModeling_29_7.png}
    \end{center}
    { \hspace*{\fill} \\}

    \newpage
    Learning rate = 0.15

    \begin{center}
    \adjustimage{max size={0.9\linewidth}{0.9\paperheight}}{DeepBeliefModeling_files/DeepBeliefModeling_29_9.png}
    \end{center}
    { \hspace*{\fill} \\}

    \newpage
    Learning rate = 0.1

    \begin{center}
    \adjustimage{max size={0.9\linewidth}{0.9\paperheight}}{DeepBeliefModeling_files/DeepBeliefModeling_29_11.png}
    \end{center}
    { \hspace*{\fill} \\}

    \newpage
    Learning rate = 0.05

    \begin{center}
    \adjustimage{max size={0.9\linewidth}{0.9\paperheight}}{DeepBeliefModeling_files/DeepBeliefModeling_29_13.png}
    \end{center}
    { \hspace*{\fill} \\}

    \newpage
    Learning rate = 0.01

    \begin{center}
    \adjustimage{max size={0.9\linewidth}{0.9\paperheight}}{DeepBeliefModeling_files/DeepBeliefModeling_29_15.png}
    \end{center}
    { \hspace*{\fill} \\}

\newpage
\subsubsection*{Osoba przekonana co do skuteczności szczepionek czyta artykuły przeciwko szczepieniom}

    Learning rate = 1.0

    \begin{center}
    \adjustimage{max size={0.9\linewidth}{0.9\paperheight}}{DeepBeliefModeling_files/DeepBeliefModeling_30_3.png}
    \end{center}
    { \hspace*{\fill} \\}

    \newpage
    Learning rate = 0.5

    \begin{center}
    \adjustimage{max size={0.9\linewidth}{0.9\paperheight}}{DeepBeliefModeling_files/DeepBeliefModeling_30_5.png}
    \end{center}
    { \hspace*{\fill} \\}

    \newpage
    Learning rate = 0.25

    \begin{center}
    \adjustimage{max size={0.9\linewidth}{0.9\paperheight}}{DeepBeliefModeling_files/DeepBeliefModeling_30_7.png}
    \end{center}
    { \hspace*{\fill} \\}

    \newpage
    Learning rate = 0.15

    \begin{center}
    \adjustimage{max size={0.9\linewidth}{0.9\paperheight}}{DeepBeliefModeling_files/DeepBeliefModeling_30_9.png}
    \end{center}
    { \hspace*{\fill} \\}

    \newpage
    Learning rate = 0.1

    \begin{center}
    \adjustimage{max size={0.9\linewidth}{0.9\paperheight}}{DeepBeliefModeling_files/DeepBeliefModeling_30_11.png}
    \end{center}
    { \hspace*{\fill} \\}

    \newpage
    Learning rate = 0.05

    \begin{center}
    \adjustimage{max size={0.9\linewidth}{0.9\paperheight}}{DeepBeliefModeling_files/DeepBeliefModeling_30_13.png}
    \end{center}
    { \hspace*{\fill} \\}

    \newpage
    Learning rate = 0.01

    \begin{center}
    \adjustimage{max size={0.9\linewidth}{0.9\paperheight}}{DeepBeliefModeling_files/DeepBeliefModeling_30_15.png}
    \end{center}
    { \hspace*{\fill} \\}

\newpage
\subsubsection*{Osoba niezdecydowana czyta artykuły zachęcające do szczepień}

    Learning rate = 1.0

    \begin{center}
    \adjustimage{max size={0.9\linewidth}{0.9\paperheight}}{DeepBeliefModeling_files/DeepBeliefModeling_31_3.png}
    \end{center}
    { \hspace*{\fill} \\}

    \newpage
    Learning rate = 0.5

    \begin{center}
    \adjustimage{max size={0.9\linewidth}{0.9\paperheight}}{DeepBeliefModeling_files/DeepBeliefModeling_31_5.png}
    \end{center}
    { \hspace*{\fill} \\}

    \newpage
    Learning rate = 0.25

    \begin{center}
    \adjustimage{max size={0.9\linewidth}{0.9\paperheight}}{DeepBeliefModeling_files/DeepBeliefModeling_31_7.png}
    \end{center}
    { \hspace*{\fill} \\}

    \newpage
    Learning rate = 0.15

    \begin{center}
    \adjustimage{max size={0.9\linewidth}{0.9\paperheight}}{DeepBeliefModeling_files/DeepBeliefModeling_31_9.png}
    \end{center}
    { \hspace*{\fill} \\}

    \newpage
    Learning rate = 0.1

    \begin{center}
    \adjustimage{max size={0.9\linewidth}{0.9\paperheight}}{DeepBeliefModeling_files/DeepBeliefModeling_31_11.png}
    \end{center}
    { \hspace*{\fill} \\}

    \newpage
    Learning rate = 0.05

    \begin{center}
    \adjustimage{max size={0.9\linewidth}{0.9\paperheight}}{DeepBeliefModeling_files/DeepBeliefModeling_31_13.png}
    \end{center}
    { \hspace*{\fill} \\}

    \newpage
    Learning rate = 0.01

    \begin{center}
    \adjustimage{max size={0.9\linewidth}{0.9\paperheight}}{DeepBeliefModeling_files/DeepBeliefModeling_31_15.png}
    \end{center}
    { \hspace*{\fill} \\}

\newpage
\subsubsection*{Osoba niezdecydowana zaczyna czytać artykuły przeciwko szczepieniom}

    Learning rate = 1.0

    \begin{center}
    \adjustimage{max size={0.9\linewidth}{0.9\paperheight}}{DeepBeliefModeling_files/DeepBeliefModeling_32_1.png}
    \end{center}
    { \hspace*{\fill} \\}

    \newpage
    Learning rate = 0.5

    \begin{center}
    \adjustimage{max size={0.9\linewidth}{0.9\paperheight}}{DeepBeliefModeling_files/DeepBeliefModeling_32_3.png}
    \end{center}
    { \hspace*{\fill} \\}

    \newpage
    Learning rate = 0.25

    \begin{center}
    \adjustimage{max size={0.9\linewidth}{0.9\paperheight}}{DeepBeliefModeling_files/DeepBeliefModeling_32_5.png}
    \end{center}
    { \hspace*{\fill} \\}

    \newpage
    Learning rate = 0.15

    \begin{center}
    \adjustimage{max size={0.9\linewidth}{0.9\paperheight}}{DeepBeliefModeling_files/DeepBeliefModeling_32_7.png}
    \end{center}
    { \hspace*{\fill} \\}

    \newpage
    Learning rate = 0.1

    \begin{center}
    \adjustimage{max size={0.9\linewidth}{0.9\paperheight}}{DeepBeliefModeling_files/DeepBeliefModeling_32_9.png}
    \end{center}
    { \hspace*{\fill} \\}

    \newpage
    Learning rate = 0.05

    \begin{center}
    \adjustimage{max size={0.9\linewidth}{0.9\paperheight}}{DeepBeliefModeling_files/DeepBeliefModeling_32_11.png}
    \end{center}
    { \hspace*{\fill} \\}

    \newpage
    Learning rate = 0.01

    \begin{center}
    \adjustimage{max size={0.9\linewidth}{0.9\paperheight}}{DeepBeliefModeling_files/DeepBeliefModeling_32_13.png}
    \end{center}
    { \hspace*{\fill} \\}
    \newpage

    \section*{Analiza i wnioski}

    Analizując powyższe wykresy, można zauważyć jak learning rate wpływa na zmianę opinii badanej osoby. Zbadane wartości współczynnika to 0.01, 0.05, 0.1, 0.15, 0.25, 0.5 oraz 1. 

Widzimy, że dla wartości 0.01, po nakarmieniu badanego artykułami z przeciwnej grupy, lub w przypadku niezdecydowanego obojętnie jakimi artykułami, słowa kluczowe zbierają się coraz bliżej siebie. Sugeruje to brak zmiany decyzji badanej osoby. 

Dla współczynnika 0.05 można zauważyć oddalanie się słów kluczowych i zbieranie w obrębie swoich grup. Można to interpretować jako świadome rozpoznawanie podziału idei, gdzie osoba czytając poznaje coraz to bardziej znaczące cechy dla danych grup.

W przypadku współczynnika równego 0.1 lub 0.15 najlepiej widać wpływ uczenia na zmianę stanowiska badanego. Można zauważyć jak słowa kluczowe z klasy przeciwnej zbliżyły się do głównego miejsca ugrupowania. Taki efekt nastąpił jednak tylko w sytuacjach gdy osoba była początkowo po którejś ze stron. Dla osób niezdecydowanych można jedynie zauważyć efekt podobny do poprzedniego punktu.

Przy współczynniku większym lub równym 0.25 tracimy możliwość jednoznacznego odczytania zmian na wykresach. Cechy grupują się, jednak rozprzestrzenione są po całej mapie. Wnioskując, współczynniki równe lub powyżej 0.25 stanowią za dużą podatność na zmianę stanowiska badanego.

Podsumowując powyższą analizę, współczynnik uczenia, tzw. learning rate, w samoorganizującej się sieci neuronowej odwzorowuje skłonność człowieka do zmiany swojej opinii przez dostęp do przekonywującej informacji. Dla najmniejszych badanych wartości osoba trzymała się swojego przekonania nie zważając na treść czytanych artykułów. Natomiast dla większych, osobę można uznać za podatną na zmiany pod wpływem otrzymywanych informacji, gdzie wartością graniczną współczynnika uczenia okazała się wartość 0.15.


\end{document}
